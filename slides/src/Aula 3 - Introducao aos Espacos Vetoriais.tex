\documentclass{beamer}
\usetheme[logo=brasaoUFSC.png]{fibeamer}
%% These macros specify information about the presentation
\title{Aula 3: Introdução aos Espaços Vetoriais}
%\subtitle{2019.2}
\author{Melissa Weber Mendonça}
%% These additional packages are used within the document:
\usepackage{ragged2e}  % `\justifying` text
\usepackage{booktabs}  % Tables
\usepackage{tabularx}
\usepackage{tikz}      % Diagrams
\usetikzlibrary{calc, shapes, backgrounds, arrows}
\usepackage{pgfplots}
\usepackage{amsmath, amssymb}
\usepackage{url}       % `\url`s
\usepackage{listings}  % Code listings
\usepackage{verbatim}
\usepackage{animate}
\usepackage{hyperref}
\hypersetup{
    colorlinks = true,
    allcolors = [rgb]{1,0.83,0.39}
}
\frenchspacing

\makeatletter
\setlength\fibeamer@lengths@logowidth{4em}
\setlength\fibeamer@lengths@logoheight{5em}
\makeatother

\newcommand{\zerodisplayskips}{%
  \setlength{\abovedisplayskip}{0pt}%
  \setlength{\belowdisplayskip}{0pt}%
  \setlength{\abovedisplayshortskip}{0pt}%
  \setlength{\belowdisplayshortskip}{0pt}}

\begin{document}
\frame{\maketitle}

\begin{darkframes}

\begin{frame}{Introdução aos espaços vetoriais}
\begin{center}
\alert{A Álgebra Linear é o estudo dos espaços vetoriais sobre corpos arbitrários e das transformações lineares entre esses espaços.}
\end{center}
\end{frame}

\begin{frame}{}
\begin{block}{Definição}
{\footnotesize{%
  Um conjunto não-vazio ${\mathbb{K}}$ é um \alert{corpo} se em ${\mathbb{K}}$ pudermos definir duas operações, denotadas por $+$ (soma) e $\cdot$ (multiplicação), satisfazendo:
  \begin{itemize}
  \setlength\itemsep{0em}
  \item[(i)] $a+b=b+a$, $\forall a, b \in {\mathbb{K}}$ (comutativa)
  \item[(ii)] $a+(b+c) = (a+b)+c$, $\forall a, b, c \in {\mathbb{K}}$ (associativa)
  \item[(iii)] Existe um elemento em ${\mathbb{K}}$, denotado por $0$ e chamado de \emph{elemento neutro da soma}, que satisfaz $0+a=a+0=a$, $\forall a \in {\mathbb{K}}$.
  \item[(iv)] Para cada $a\in {\mathbb{K}}$, existe um elemento em ${\mathbb{K}}$ denotado por $-a$ e chamado de \emph{oposto de $a$} (ou \emph{inverso aditivo de $a$}) tal que $a+(-a)=(-a)+a = 0$.
  \item[(v)] $a\cdot b = b\cdot a$, $\forall a, b \in {\mathbb{K}}$ (comutativa)
  \item[(vi)] $a\cdot(b\cdot c) = (a\cdot b)\cdot c$, $\forall a, b, c \in {\mathbb{K}}$
  \item[(vii)] Existe um elemento em ${\mathbb{K}}$ denotado por $1$ e chamado de \emph{elemento neutro da multiplicação}, tal que $1\cdot a = a\cdot 1 = a$, $\forall a \in {\mathbb{K}}$.
  \item[(viii)] Para cada elemento não-nulo $a\in {\mathbb{K}}$, existe um elemento em ${\mathbb{K}}$, denotado por $a^{-1}$ e chamado de \emph{inverso multiplicativo de $a$}, tal que $a\cdot a^{-1}=a^{-1}\cdot a = 1$.
  \item[(ix)] $(a+b)\cdot c = a\cdot c + b\cdot c$, $\forall a, b, c \in {\mathbb{K}}$ (distributiva).
  \end{itemize}
  }}
\end{block}
\end{frame}

\begin{frame}{Exemplos de corpos}
  São corpos: 
  \begin{itemize}
      \item ${\mathbb{Q}}$
      \item ${\mathbb{R}}$
      \item ${\mathbb{C}}$
  \end{itemize} 
\end{frame}

\begin{frame}{}
\begin{block}{Definição}
{\footnotesize{%
  Um conjunto não vazio $E$ é um \alert{espaço vetorial} sobre um corpo ${\mathbb{K}}$ se em seus elementos, denominados \emph{vetores}, estiverem definidas duas operações: 
  \begin{itemize}
    \setlength\itemsep{0em}
  \item \emph{soma}: A cada $u,v \in E$, associa $u+v \in E$
  \item \emph{multiplicação por um escalar}: a cada escalar $\alpha \in {\mathbb{K}}$ e a cada vetor $v \in E$, associa $\alpha v \in E$.
  \end{itemize}
  Estas operações devem satisfazer as condições abaixo:
  \begin{itemize}
    \setlength\itemsep{0em}
  \item[(i)] {\bf{Comutatividade:}} $u+v = v+u$, $\forall u, v \in E$
  \item[(ii)] {\bf{Associatividade:}} $(u+v)+w = u+(v+w)$ e $(\alpha \beta)v = \alpha (\beta v)$, $\forall u, v, w\in E$ e $\alpha, \beta \in {\mathbb{K}}$
  \item[(iii)] {\bf{Existência do vetor nulo:}} existe um vetor $0\in E$, chamado \emph{vetor nulo}, tal que $v+0 = 0+v = v$ para todo $v\in E$.
  \item[(iv)] {\bf{Existência do inverso aditivo:}} para cada vetor $v\in E$ existe um vetor $-v \in E$ chamado \emph{inverso aditivo} tal que $-v+v = v+(-v) = 0\in E$.
  \item[(v)] {\bf{Distributividade:}} $(\alpha + \beta)v = \alpha v + \beta v$ e $\alpha(u+v) = \alpha u + \alpha v$, $\forall u, v \in E$ e $\forall \alpha, \beta \in {\mathbb{K}}$
  \item[(vi)] {\bf{Multiplicação por 1:}} $1\cdot v = v$, em que $1$ é o elemento neutro da multiplicação em ${\mathbb{K}}$.
  \end{itemize}
  }}
\end{block}
\end{frame}

\begin{frame}{Exemplos}
  \only<1-2>{\begin{itemize}
    \item Todo corpo é um espaço vetorial sobre si mesmo. 
    \end{itemize}
  \only<2>{\vfill De fato, se ${\mathbb{K}}$ é um corpo, então as duas operações internas em ${\mathbb{K}}$ podem ser vistas como a soma de vetores e a multiplicação por escalares.}}
  \only<3>{\begin{itemize}
    \item Para todo número natural $n$, o conjunto ${\mathbb{K}}^n$, definido como
      \begin{equation*}
        \zerodisplayskips
        {\mathbb{K}}^n = {\mathbb{K}} \times \cdots \times {\mathbb{K}} = \{ (u_1,\ldots,u_n) : u_i \in {\mathbb{K}}, \forall i = 1,\ldots, n\}
      \end{equation*}
      é um espaço vetorial sobre ${\mathbb{K}}$.
    \end{itemize}}
  
  \only<4>{\begin{itemize}
    \item Os elementos do espaço vetorial ${\mathbb{R}}^{\infty}$ são as sequências infinitas de números reais do tipo
      \begin{equation*}
        \zerodisplayskips
        u = (\alpha_1,\ldots,\alpha_n,\ldots).
      \end{equation*}
      \begin{itemize}
      \item O elemento zero é a sequência formada por infinitos zeros $0=(0,\ldots,0,\ldots)$;
      \item O inverso aditivo da sequência $u$ é $-u=(-\alpha_1,\ldots,-\alpha_n,\ldots)$;
      \item As operações de adição e multiplicação por escalar são definidas por
        \begin{align*}
          u+v &= (\alpha_1+\beta_1,\ldots,\alpha_n+\beta_n,\ldots)\\
          \rho u &= (\rho \alpha_1,\ldots,\rho \alpha_n,\ldots).
        \end{align*}
      \end{itemize}
    \end{itemize}
  }

  \only<5-6>{\begin{itemize}
    \item O conjunto ${\mathcal{M}}_{m\times n}({\mathbb{K}})$ de todas as matrizes $m\times n$ com elementos em ${\mathbb{K}}$ é um espaço vetorial?
    \end{itemize}

    \only<6>{{\footnotesize{Sim, se
      \begin{itemize}
      \item A soma entre duas matrizes $A=[a_{ij}]$ e $B=[b_{ij}]$ é dada por
        \begin{equation*}
          [A+B]_{ij} = a_{ij}+b_{ij}
        \end{equation*}
      \item O produto de uma matriz $A$ pelo escalar $\rho \in {\mathbb{K}}$ como 
        \begin{equation*}
          [\rho A]_{ij} = \rho a_{ij}
        \end{equation*}
      \item A matriz nula $0 \in {\mathcal{M}}_{m\times n}$ é aquela formada por zeros;
      \item O inverso aditivo da matriz $A=[a_{ij}]$ é a matriz $-A =[-a_{ij}]$.
    \end{itemize}
  }}}}
  \only<7>{\begin{itemize}
    \item  O conjunto de polinômios
      \begin{equation*}
        {\mathcal{P}}({\mathbb{K}}) = \left \{ p(x) = a_nx^n+\ldots+a_1x+a_0 : a_i \in {\mathbb{K}} \text{ e } n\geq 0\right \}
      \end{equation*}
      é um espaço vetorial com as operações usuais de soma de polinômios e multiplicação por escalar.
    \end{itemize}
  }
  \only<8>{\begin{itemize}
    \item Seja $X$ um conjunto não-vazio qualquer. O símbolo ${\mathcal{F}}(X;{\mathbb{R}})$ representa o conjunto de todas as funções reais $f:X\rightarrow {\mathbb{R}}$. Esse conjunto é um espaço vetorial. 
    \end{itemize}
    \vfill
    Variando o conjunto $X$, obtemos:
    \begin{itemize}
    \item Se $X = \{1,\ldots,n\}$, então ${\mathcal{F}}(X;{\mathbb{R}}) = {\mathbb{R}}^n$, pois a cada número em $X$ associamos um número real $\alpha$, gerando assim uma lista de $n$ valores reais para cada elemento do conjunto.
    \item Se $X={\mathbb{N}}$, então ${\mathcal{F}}(X;{\mathbb{R}}) = {\mathbb{R}}^{\infty}$.
    \item Se $X$ é o produto cartesiano dos conjuntos $\{1,\ldots,m\}$ e $\{1,\ldots,n\}$ então ${\mathcal{F}}(X;{\mathbb{R}}) = {\mathcal{M}}_{m\times n}$.
    \end{itemize}
    }
\end{frame}

\begin{frame}{Propriedades}
Como consequência dos axiomas, valem num espaço vetorial as regras operacionais habitualmente usadas nas manipulações numéricas:
\begin{enumerate}
\item Para todos $u,v,w \in E$, temos que $w+u = w+v \Rightarrow u=v$. Em particular, $w+u = w \Rightarrow u=0$ e $w+u = 0 \Rightarrow u=-w$.
\item Dados $0\in {\mathbb{K}}$ e $v\in E$, temos que $0v = 0 \in E$. Analogamente, dados $\alpha \in {\mathbb{K}}$ e $0\in E$, temos que $\alpha 0 = 0$.
\item Se $\alpha \ne 0$ e $v\ne 0$ então $\alpha v \ne 0$.
\item $(-1)v = -v$.
\end{enumerate}
\end{frame}

\begin{frame}{Observação}
Um espaço vetorial sobre um corpo ${\mathbb{K}}$ é um conjunto $E$ de \emph{vetores}, com uma operação de soma que é uma função $+:E\to E$ e uma operação de produto por escalar, que é uma função $\cdot: {\mathbb{K}}\times E:E$, satisfazendo os axiomas listados acima. Note que os axiomas não involvem a propriedade de inverso multiplicativo do corpo, e podemos definir uma estrutura semelhante à de espaço vetorial sobre um anel, que chamamos de \emph{módulo} sobre ${\mathbb{K}}$. No entanto, a maioria dos teoremas provados para espaços vetoriais não seria válida nos módulos; por exemplo, não podemos falar da dimensão de um módulo.
\end{frame}

\begin{frame}{Subespaços vetoriais}

\begin{block}{Definição}
  Um subespaço vetorial do espaço vetorial $E$ é um subconjunto $F\subset E$ que, relativamente às operações de $E$, é ainda um espaço vetorial, ou seja, satisfaz 
  \begin{itemize}
  \item[(i)] Para todo $u,v \in F$, $u+v \in F$
  \item[(ii)] Para todo $u\in F$ e $\alpha \in {\mathbb{K}}$, $\alpha u \in F$.
  \end{itemize}
\end{block}
\end{frame}

\begin{frame}{Observações}
  \only<1>{\begin{itemize}
    \item Note que no caso de um subespaço, não é necessário verificar as seis propriedades listadas anteriormente pois elas já são satisfeitas para $E$, e $F \subset E$. No entanto, um subespaço deve ser \emph{fechado} para a adição e a multiplicação por escalar. Mais geralmente, dados $v_1,\ldots,v_m \in F$ e $\alpha_1,\ldots,\alpha_m \in {\mathbb{K}}$, 
      \begin{equation*}
        \zerodisplayskips
        v = \alpha_1v_1+\ldots+\alpha_mv_m
      \end{equation*}
      deve pertencer a $F$.
    \end{itemize}}
  \only<2>{\begin{itemize}
    \item O vetor nulo pertence a \emph{todos} os subespaços.
    \end{itemize}}
  \only<3>{\begin{itemize}
    \item O espaço inteiro $E$ é um exemplo trivial de subespaço de $E$.
    \end{itemize}}
  \only<4>{\begin{itemize}
    \item Todo subespaço é, em si mesmo, um espaço vetorial.
    \end{itemize}}
  \only<5>{\begin{itemize}
    \item O conjunto vazio não pode ser um subespaço vetorial.
    \end{itemize}}
\end{frame}

\begin{frame}{Exemplos}
  \only<1>{\begin{itemize}
    \item Seja $v\in E$ um vetor não-nulo. O conjunto $F = \{\alpha v: \alpha \in {\mathbb{K}}\}$ de todos os múltiplos de $v$ é um subespaço vetorial de $E$, chamado de \emph{reta que passa pela origem e contém $v$}.
    \end{itemize}}
  \only<2>{\begin{itemize}
    \item Seja $E={\mathcal{F}}({\mathbb{R}};{\mathbb{R}})$ o espaço vetorial das funções reais de uma variável real $f:{\mathbb{R}} \rightarrow {\mathbb{R}}$. Para cada $k\in {\mathbb{N}}$, o conjunto ${\mathcal{C}}^k ({\mathbb{R}})$ das funções $k$ vezes continuamente diferenciáveis é um subespaço vetorial de $E$.
    \end{itemize}}
  \only<3>{\begin{itemize}
    \item Sejam $a_1,\ldots,a_n$ números reais. O conjunto ${\mathcal{H}}$ de todos os vetores $v=(x_1,\ldots,x_n) \in {\mathbb{R}}^n$ tais que
      \begin{equation*}
        a_1x_1+\ldots+a_nx_n=0
      \end{equation*}
      é um subespaço vetorial de ${\mathbb{R}}^n$. No caso trivial em que $a_1=\ldots=a_n=0$, o subespaço ${\mathcal{H}}$ é todo o ${\mathbb{R}}^n$. Se, ao contrário, pelo menos um dos $a_i\ne 0$, ${\mathcal{H}}$ chama-se \emph{hiperplano} de ${\mathbb{R}}^n$ que passa pela origem.
    \end{itemize}}
  \only<4>{\begin{itemize}
    \item Seja $E$ o espaço das matrizes $3\times 3$: $E = \{ A \in {\mathbb{R}}^{3\times 3}\}$. O conjunto das matrizes triangulares inferiores de dimensão 3 é um subespaço de $E$, assim como o conjunto das matrizes simétricas.
    \end{itemize}}
  \only<5>{\begin{itemize}
    \item Dentro do espaço ${\mathbb{R}}^3$, os subespaços possíveis são: o subespaço nulo, o espaço inteiro, as retas que passam pela origem, e os planos que passam pela origem. Qualquer reta que não passe pela origem não pode ser um subespaço (pois não contem o vetor nulo).
    \end{itemize}}
\end{frame}

\begin{frame}
  \begin{block}{Teorema}
    Dados um espaço vetorial $E$ e subespaços $F_1,F_2 \subset E$, a interseção $F_1 \cap F_2$ ainda é um subespaço de $E$.
  \end{block}
\end{frame}

% \begin{proof}
% Primeiramente, note que $F_1\cap F_2$ nunca é vazio, pois $0\in F_1$ e $0\in F_2$. Precisamos então verificar as duas condições que definem um subespaço vetorial.
% \begin{itemize}
% \item[(i)] Sejam $u,v \in F_1\cap F_2$. Então, $u,v \in F_1$ e $u,v\in F_2$. Logo, como $F_1$ e $F_2$ são ambos subespaços de $E$, $u+v\in F_1$ e $u+v \in F_2$, portanto $u+v\in F_1 \cap F_2$.
% \item[(ii)] Seja $u\in F_1\cap F_2$ e $\alpha \in {\mathbb{K}}$. Então, $u\in F_1$ e $u\in F_2$. Como ambos $F_1$ e $F_2$ são subespaços de $E$, $\alpha u \in F_1$ e $\alpha u \in F_2$. Portanto, $\alpha u \in F_1 \cap F_2$.
% \end{itemize}
% Assim, provamos que a interseção dos dois subespaços é também um subespaço vetorial de $E$.
% \end{proof}

\begin{frame}{E a união?}
  A união de dois subespaços vetoriais \emph{não} é (em geral) um subespaço vetorial. 

  \vfill
  \only<2>{%
    Contra-exemplo:
    \begin{align*}
      E &= {\mathbb{R}}^{n\times n}\\
      F_1 &= \{ \text{ matrizes triangulares superiores }\}\\
      F_2 &= \{ \text{ matrizes triangulares inferiores }\}
    \end{align*}
    \begin{itemize}
    \item O que é $F_1\cap F_2$?
    \item O que é $F_1\cup F_2$?
    \end{itemize}}
\end{frame}

\begin{frame}{Exemplo}
  $E = {\mathbb{R}}^3$, $F_1,F_2$ dois planos em ${\mathbb{R}}^3$ passando pela origem.
  \begin{itemize}
  \item $F_1\cap F_2$ é a reta de interseção de $F_1$ e $F_2$ passando pela origem;
  \item $F_1\cup F_2$ é a união dos dois planos.
  \end{itemize}
  
  \begin{center}
    \begin{tikzpicture}[axis/.style={thick, ->, >=stealth'}, scale=0.8]
      \draw[axis] (0,0) -- (0,3);
      \draw[axis] (3,0.75) -- (4,1);
      \draw[axis] (0,0) -- (2.5,-2.5);
      \draw[thick] (0,0) -- (0,2) -- (3,2) -- (3,0) -- (0,0);
      \draw[thick, fill=orange!90] (0,0) -- (0,2) -- (2.5,1) -- (2.5,-1) -- (0,0);
      \draw[thick, dashed] (0,0) -- (3,0.75);
      \draw[thick, dashed] (0,0) -- (2.5,0);
      \node[anchor=west] at (1.7,0.8) {$F_1$};
      \node[anchor=west] at (2.3,1.6) {$F_2$};
      % 
      \draw[very thick] (0,0) -- (0,2);
      \node[anchor=west] at (-1.7,1) {$F_1\cap F_2$};
      % 
      \draw[thick, fill=orange!90] (6,0) -- (6,2) -- (9,2) -- (9,0) -- (6,0);
      \draw[thick, fill=orange!90] (6,0) -- (6,2) -- (8.5,1) -- (8.5,-1) -- (6,0);
      \draw[thick, dashed] (6,0) -- (8.5,0);
      \node[anchor=west] at (7.7,0.8) {$F_1$};
      \node[anchor=west] at (8.3,1.6) {$F_2$};
      \node[anchor=west] at (9,1) {$F_1\cup F_2$};
    \end{tikzpicture}
  \end{center}
\end{frame}

\begin{frame}{Exemplo}
  $E={\mathbb{R}}^3$, $F_1$ e $F_2$ duas retas que passam pela origem. Ambos $F_1$ e $F_2$ são subespaços, mas sua união, representada pelo feixe das duas retas, não o é.
  \begin{center}
    \begin{tikzpicture}[axis/.style={thick, ->, >=stealth'}]
      \draw[axis] (0,0) -- (0,3);
      \draw[axis] (0,0) -- (4,1);
      \draw[axis, dashed] (0,0) -- (3,-1);
      \node[anchor=west] at (4,1) {$F_1$};
      \node[anchor=west] at (0,3) {$F_2$};
      \draw[very thick, ->, >=stealth'] (0,0) -- (2,0.5);
      \node[anchor=west] at (2,0.2) {$u$};
      \draw[very thick, ->, >=stealth'] (0,0) -- (0,2.5);
      \node[anchor=west] at (-0.5,2.5) {$v$};
      \draw[dashed] (2,0.5) -- (2,3) -- (0,2.5);
      \draw[very thick, ->, >=stealth'] (0,0) -- (2,3);
      \node[anchor=west] at (2,3) {$u+v$};
    \end{tikzpicture}
  \end{center}
\end{frame}

\begin{frame}{Será que existe alternativa?}
  Como vimos no último exemplo, a união de dois subespaços vetoriais não é necessariamente um subespaço vetorial. No entanto, podemos construir um conjunto $S$ que contém $F_1$ e $F_2$ e que é subespaço de $E$, como veremos no Teorema a seguir.
\end{frame}

\begin{frame}{}
  \begin{block}{Teorema}
  Sejam $F_1$ e $F_2$ subespaços de um espaço vetorial $E$. Então o conjunto
  \begin{equation*}
    S = F_1+F_2 = \{ w \in E : w = w_1+w_2, w_1\in F_1, w_2\in F_2\}
  \end{equation*}
  é um subespaço de $E$.
\end{block}
\vfill
\alert{Demonstração. } Vamos verificar as condições para que $S$ seja um subespaço de $E$.

\only<2>{Primeiramente, note que $0\in S$ pois $0 \in F_1$ e $0\in F_2$. }
\only<3>{%
  \begin{itemize}
  \item[(i)] Sejam $v, w \in S$. Então $v = v_1+v_2$, $v_1 \in F_1$, $v_2 \in F_2$ e $w = w_1+w_2$, $w_1 \in F_1$, $w_2 \in F_2$. Assim \vspace*{-0.5cm}
    \begin{align*}
      v+w &= (v_1+v_2)+(w_1+w_2)\\
          &= (v_1+w_1) + (v_2+w_2) \in S,
    \end{align*}
    \vskip-0.5cm
    pois $v_1+w_1 \in F_1$ e $v_2+w_2 \in F_2$ já que ambos são subespaços de $E$ e $v_1,w_1 \in F_1$ e $v_2,w_2 \in F_2$, e a última igualdade segue das propriedades da soma no espaço vetorial $E$.
  \end{itemize}}
\only<4>{%
  \begin{itemize}
  \item[(ii)] Sejam $\alpha \in {\mathbb{K}}$ e $w \in S$. Então,
    \begin{equation*}
      \alpha w = \alpha (w_1+w_2) = \alpha w_1 + \alpha w_2 \in S
    \end{equation*}
    já que $\alpha w_1 \in F_1$ e $\alpha w_2 \in F_2$ pois ambos são subespaços de $E$.
  \end{itemize}}
\end{frame}

\begin{frame}{Exemplo}
  No Exemplo da união dos subespaços, $S = F_1+F_2$ é o plano que contém as duas retas.
\end{frame}

\end{darkframes}
\end{document}

